 

\section{Elements of the 2D PIC-MCC simulations}
  \label{sec-elements}
  \subsection{The PIC simulations}
    \label{subsec-intro}
    The \ac{PIC} simulation models particles moving freely on a grid.
    The grid is used to compute the electric field, in the electrostatic approximation by solving the Poisson equation
    \begin{equation}
      \label{eq-poisson}
      \Delta \phi = - \frac{\rho}{\epsilon_0}
    \end{equation}
    where $\phi$ is the electric potential, $\rho$ is the charge density, and $\epsilon_0$ the vacuum permittivity.
    If the electrostatic approximation is not correct, one needs to solve the Maxwell equations.

    The particles move following the Lorentz forces
    \begin{equation}
      \label{eq-Lor}
      m \deriv{\vec{v}}{t} = q \vect{E} + q \vec{v} \times \vec{B}
    \end{equation}
    with $m$ and $q$, the particle mass and electric charge, respectively.
    The numerical particles followed in the simulations correspond to $q_f$ physical particles, with
    \begin{equation}
      q_f = \frac{n V}{\Npc}
    \end{equation}
    with $n$ the particle density, $V$ the volume of a cell, and $\Npc$ the number of numerical particles in a cell.
    A large enough number of particles is needed in order to obtain physical results.
    Indeed, an insufficient number of particles leads to numerical heating \cite{ueda1994}.
    Usually, a minimum of 100 particles per cell is used, but recent results seem to encourage to use more particles \cite{janhunen2018}.

  \subsection{The Monte Carlo collisions}

    In \ac{PIC} simulations, collisions between charged and neutral particles can be modeled by binary collision, but this approach is computationally costly.
    Instead, a Monte-Carlo algorithm can be used \cite{vahedi1995}.
    This approach is very efficient and allows scattering, momentum transfer, and ionization to be consistently modeled.
    The propellant used in \ac{HET} is \ac{Xe}.
    The cross-sections used for modeling \ac{Xe} or other gases collisions are taken from the {\sc LXCat} database project \cite{LXCat_web,pancheshnyi2012}.
    Unless otherwise stated, the elastic, inelastic scattering and ionization reactions listed in \cref{tab-reactXe} are used.
    The cross-section values are summarized in \cref{fig-xexsection}.

    \begin{table}[hbt]
      \ra{1.3}
      \centering
      \caption{Reactions for xenon used in the PIC simulations}
      \label{tab-reactXe}
      \begin{tabular}{@{}lll@{}}  \toprule
        Reaction & Threshold & Reference\\ \midrule
        {\it Elastic scattering} & &\\
        e + Xe = e + Xe   & --   & \cite{Lxcat_Xe,Lxcat_Xe2} \\
        {\it Excitation} & &\\
        e + Xe = e + Xe$^*$   & 8.315eV   & \cite{Lxcat_Xe,Lxcat_Xe2} \\
        e + Xe = e + Xe$^*$   & 9.447eV   & \cite{Lxcat_Xe,Lxcat_Xe2} \\
        e + Xe = e + Xe$^*$   & 9.917eV   & \cite{Lxcat_Xe,Lxcat_Xe2} \\
        e + Xe = e + Xe$^*$   & 11.7eV    & \cite{Lxcat_Xe,Lxcat_Xe2} \\
        {\it Ionization} & &\\
        e + Xe = e + Xe$^+$   & 12.13eV   & \cite{Lxcat_Xe,Lxcat_Xe2} \\
        \bottomrule
      \end{tabular}
    \end{table}



    \begin{figure}[hbt]
      \centering
      \includegraphics[width=\defaultwidth]{figure/xenon_cross_section.pdf}
      \caption{Cross section values used in the Monte Carlo procedure \cite{Lxcat_Xe,Lxcat_Xe2}.}
      \label{fig-xexsection}
    \end{figure}

    In the context of this thesis, and except precised otherwise, the `\ac{PIC} simulation' refers to the `\ac{PIC}-\ac{MCC} simulation'.
    In the case where no collision is modeled, we also call it `collisionless simulation'.
